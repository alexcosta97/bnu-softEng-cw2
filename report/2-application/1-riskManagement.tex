\chapter{Risk Management}
\section{Risk Identification}
For the FASAM, considering the current team and the type of project, the following risks have been identified:
\begin{itemize}
    \item Estimation risks
    \begin{enumerate}
        \item The time required to develop the software was underestimated
        \item The size of the software was underestimated
    \end{enumerate}
    \item Organizational risks
    \begin{enumerate}
        \item The organization is restructured
        \item There is a cut in the project budget
    \end{enumerate}
    \item People risks
    \begin{enumerate}
        \item Key staff leaves
        \item There is not enough skilled staff
        \item THere is no training available for staff
    \end{enumerate}
    \item Requirements risks
    \begin{enumerate}
        \item Changes proposed require major re-design of the system
        \item The legislation changes
        \item Customers don't understand the impact of changes to the system
    \end{enumerate}
    \item Technology risks
    \begin{enumerate}
        \item Faults in the communication interfaces with the hardware require reparation before being usable
        \item The technology chosen cannot cope with the performance needs of a real-time system
    \end{enumerate}
    \item Tools
    \begin{enumerate}
        \item The code generated by software code generation tools is inefficient
        \item The different software tools don't work together efficiently
    \end{enumerate}
\end{itemize}

\section{Risk Analysis}
After analyzing and studying the identified risks, here is the probability and effects identified for each of them:

\begin{itemize}
    \item Estimation risks
    \begin{enumerate}
        \item High probability; Serious effects
        \item High probability; Tolerable effects
    \end{enumerate}
    \item Organizational risks
    \begin{enumerate}
        \item Low probability; Serious effects
        \item Low probability; Catastrophic effects
    \end{enumerate}
    \item People's risks
    \begin{enumerate}
        \item Medium probability; Serious effects
        \item Low probability; Catastrophic effects
        \item Medium probability; Tolerable effects
    \end{enumerate}
    \item Requirements risks
    \begin{enumerate}
        \item Medium probability; Serious effects
        \item Low probability; Catastrphic effects
        \item Medium probability; Tolerable effects
    \end{enumerate}
    \item Technology risks
    \begin{enumerate}
        \item Low probability; Serious effects
        \item Medium probability; Serious effects
    \end{enumerate}
    \item Tools risks
    \begin{enumerate}
        \item Medium probability; Insignificant effects
        \item High Probability; Tolerable effects
    \end{enumerate}
\end{itemize}

After knowing this, we were able to prioritize the order in which to address the risks. To do that, we prioritized first the risks with the most effect, and, inside the group of risks with the same level of effects, we prioritized the one with the most likelihood of occuring, which leads us to the following list of risks to address:

\begin{enumerate}
    \item There is not enough skilled staff
    \item Legislation changes
    \item There is a cut in the project budget
    \item The time required to develop the software was underestimated
    \item Key staff leaves
    \item Changes proposed require major re-design
    \item The technology chosen cannot cope with the performance needs of a real-time system
    \item Faults in the communication interfaces with the hardware require reparation before being usable
    \item The organization is restructured
    \item The size of the software was underestimated
    \item The different software tools don't work together efficiently
    \item Customers don't understand the impact of changes to the system
    \item There is no training available for staff
    \item The code generated by software code generation tools is inefficient
\end{enumerate}

\section{Risk Planning}
Considering that FASAM is a large-scale, safety-critical and real-time software certain aspects need to be taken more into account than in other types of projects. For example, delivering a software that is compliant with the current legislation and made by skilled people is very important, otherwise the security of people could be put at risk. With that in mind, the follwiing strategies have been devised to avoid the risks or minimize their effect. The list of strategies follows the numbering used for the priotization of the risks, therefore indicating that the strategy is to be applied for that risk:

\begin{enumerate}
    \item Avoidance: Plan needed stagg in advance; Minimizing: Hire and train staff as necessary
    \item Avoidance: Prepare for change in advance if changes are previewed. Minimizing: Keep updated about possible legislation changes
    \item Avoidance: Listen to company rumors and plan the budget wisely; Contigency: Prepare to pitch the organization about the importance of keepping the budget
    \item Avoidance: Give stakeholders a timeframe bigger than originally thought of about three weeks and give more time than planned for staff to complete tasks; Minimizing: Keep stakeholders up-to-date with the finish date
    \item Avoidance: Make sure staff is happy; Minimizing: Train other people to take over position
    \item Avoidance: Make software desigfn moduler for easier change; Minimizing: Talk to customers about the implications of change, plan more time in the timeframe for change
    \item Avoidance: Study  technologies carefully before choosing the ones for the project and hire experts in the technology chosen.
    \item Avoidance: Choose hardware that has good feedback in similar setups and choose carefully the interfaces to use to communicate with them
    \item Minimizing: Make sure everything is documented and at least one member of staff knows all the ins and outs of the project to put someone else up to speed quickly
    \item Avoidance: Analyse the requirements carefully and thoroughly design the system to give the development and management team a good idea of the size of the project. Minimizing: Make staff work extra hours and give the customer thorough explanations on why it is taking more time to finish the project.
    \item Avoidance: Plan to use tools that communicate well together. Minimizing: Make sure all staff is trained on how to carry over information between the incompatible systems.
    \item Avoidance: Explain to customers the possible impact of changes; Minimizing: Always give the customers a revised timeframe and budget for the requested changes.
    \item Avoidance: Choose widely used technologies; Minimizing: Hire staff experienced in that particular technology
    \item Avoidance: Don't use code generation; Minimizing: Get experienced staff to verify the performance and efficiency of the auto-generated code and optimize it if necessary.
\end{enumerate}

Those plans aren't the best, but give an idea of what to do in order to avoid and cope with the different risks and problems that may arise during the development of the software. They may need to be revised or different actions might have to be taken when monitoring the risks and as the development goes.

\section{Risk Monitoring}
Throughout the project, some risks happened and, because of our risk management strategy, we managed to deal with them effectively, allowing us to deliver the software without much delay and change in budget.

The following risks and ways to deal with them occured:
\begin{itemize}
    \item The time required to develop the project was underestimated
    \begin{itemize}
        \item We took longer than we thought it would take to complete the project
        \item The risk didn't really affect the delivery of the project on time since we asked the client for more time to finish the project than we thought we would need
    \end{itemize}
    \item There is not enough skilled staff
    \begin{itemize}
        \item When we first started the project we realized that we didn't have skilled enough staff to develop the software.
        \item We hired skilled enough people before even finishing the requirements gathering and analysis, therefore avoiding the risk from taking big proportions
    \end{itemize}
    \item Customers don't understand the impact of change
    \begin{itemize}
        \item Clients often asked us for changes that would require major redesign and even sometimes go against safety legislations
        \item We had to negotiate with them and compromise on the changes they wanted to be made in order for the software to be developed on time and to comply to all health and safety legislations
    \end{itemize}
\end{itemize}

Overall, not that many risks occured during the development, which we are grateful for, but it was still good to have plans to avoid most of the risks, and contigency and minimizing plans should the risks still occur.