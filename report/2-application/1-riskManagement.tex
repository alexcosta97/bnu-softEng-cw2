\chapter{Risk Management}
\section{Risk Identification}
For the FASAM, considering the current team and the type of project, the following risks have been identified:
\begin{itemize}
    \item Estimation risks
    \begin{enumerate}
        \item The time required to develop the software was underestimated
        \item The size of the software was underestimated
    \end{enumerate}
    \item Organizational risks
    \begin{enumerate}
        \item The organization is restructured
        \item There is a cut in the project budget
    \end{enumerate}
    \item People risks
    \begin{enumerate}
        \item Key staff leaves
        \item There is not enough skilled staff
        \item THere is no training available for staff
    \end{enumerate}
    \item Requirements risks
    \begin{enumerate}
        \item Changes proposed require major re-design of the system
        \item The legislation changes
        \item Customers don't understand the impact of changes to the system
    \end{enumerate}
    \item Technology risks
    \begin{enumerate}
        \item Faults in the communication interfaces with the hardware require reparation before being usable
        \item The technology chosen cannot cope with the performance needs of a real-time system
    \end{enumerate}
    \item Tools
    \begin{enumerate}
        \item The code generated by software code generation tools is inefficient
        \item The different software tools don't work together efficiently
    \end{enumerate}
\end{itemize}

\section{Risk Analysis}
After analyzing and studying the identified risks, here is the probability and effects identified for each of them:

\begin{itemize}
    \item Estimation risks
    \begin{enumerate}
        \item High probability; Serious effects
        \item High probability; Tolerable effects
    \end{enumerate}
    \item Organizational risks
    \begin{enumerate}
        \item Low probability; Serious effects
        \item Low probability; Catastrophic effects
    \end{enumerate}
    \item People's risks
    \begin{enumerate}
        \item Medium probability; Serious effects
        \item Low probability; Catastrophic effects
        \item Medium probability; Tolerable effects
    \end{enumerate}
    \item Requirements risks
    \begin{enumerate}
        \item Medium probability; Serious effects
        \item Low probability; Catastrphic effects
        \item Medium probability; Tolerable effects
    \end{enumerate}
    \item Technology risks
    \begin{enumerate}
        \item Low probability; Serious effects
        \item Medium probability; Serious effects
    \end{enumerate}
    \item Tools risks
    \begin{enumerate}
        \item Medium probability; Insignificant effects
        \item High Probability; Tolerable effects
    \end{enumerate}
\end{itemize}

After knowing this, we were able to prioritize the order in which to address the risks. To do that, we prioritized first the risks with the most effect, and, inside the group of risks with the same level of effects, we prioritized the one with the most likelihood of occuring, which leads us to the following list of risks to address:

\begin{enumerate}
    \item There is not enough skilled staff
    \item Legislation changes
    \item There is a cut in the project budget
    \item The time required to develop the software was underestimated
    \item Key staff leaves
    \item Changes proposed require major re-design
    \item The technology chosen cannot cope with the performance needs of a real-time system
    \item Faults in the communication interfaces with the hardware require reparation before being usable
    \item The organization is restructured
    \item The size of the software was underestimated
    \item The different software tools don't work together efficiently
    \item Customers don't understand the impact of changes to the system
    \item There is no training available for staff
    \item The code generated by software code generation tools is inefficient
\end{enumerate}