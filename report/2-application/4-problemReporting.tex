\chapter{Problem Reporting and Correction Startegy}
Because of the nature of the project and the methodology chosen, the approach taken to report problems and correct them during the development of the FASAM project will be of a very much formal nature. In order to report problems, members of staff will fill in a form describing the problem encountered, how the encountered it and a solution that they believe will solve the problem.

If the nature of the problem is in the code itself, that will be done thorugh a bug report in the version management system, otherwise it will join the rest of the documentation for the project.

All code related problems (meaning bugs), will be solved using the usual procedures of bug detection and correction.

After the problem has been found, all members envolved in that part of the project are to be gathered in a meeting to discuss the issue and devise a strategy to solve the problem, taking into account the possible solution written in the problem report form. Minutes of decisions made during that meeting are to be taken and added to the problem log.

Whenever a trial has been done to solve a problem, the same must be logged and the results of which must join the documentation as well.

After the problem has been solved, the problem must be declared so in the log so that all team members are aware that the problem no longer exists.

\section{Example}
During the development of FASAM, we encountered one problem with our documentation, where some requirements were missing that were to be implemented.

The person that found the problem opened a log in our management system and reported the problem to their superior, which then went and gathered the people responsible for documenting the requirements. They sat down and figured out which requirements were missing from the documentation that were present in the code, by going thorugh the version management system to find all the design processes for those requirements and the first elicitation of said requirements.

They then solved the problem by gathering all the information about those requirements and adding them to the documentation. Whilst doing that job, they added their progress to the problem log in the management system, and, when the problem was solved, they declared the problem as solved by closing the issue.