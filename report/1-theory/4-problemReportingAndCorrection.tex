\chapter{Problem Reporting and Correction}
\section{Problem Prioritising}
During the life-cycle of a project, many problems can arise, and it is necessary to be able to keep track of them and correct them as they arise.

Depending on the threat that they pose to the development of the system, problems that need to be fixed and solved need to be prioritised so that the project has minimal delay.

Prioritising of problems to solve or fix needs to be done according to the risk assessment of the project, meaning that problems that cause bigger threats to the accomplishment of the project should be solved first.

\section{Types of Problems}
Many different types of problems may arise during the development of a system. Here are some of the problems that may occur when developing a system:
\begin{enumerate}
    \item Requirement Analysis carried out incorrectly
    \item Requirements change
    \item Error in the software
    \item Performance issue of software and hardware used
    \item Delays to deliver the product
    \item Staff turnnover
    \item etc.
\end{enumerate}

\section{Problem Reporting}
Depending on the type of problem and the approach used for the development of the system, different approaches may be used to report a problem. Here are the different possible ways to report a problem:
\begin{enumerate}
    \item Formal problem report: A standardized document describing the problem, when it occured and how it occured
    \item Informal problem report: By talking to a superior or a person responsible about the problem, and communicating directly in person about it.
\end{enumerate}

In an agile approach, the most common way to report a problem, especially regarding a software error (or bug) would be through an informal problem report, whilst a waterfall approach would be more keen to use a formal problem report to report a similar problem.

In regard to software errors, errors are most often found by testing the software and by user feedback. Whenever errors are found with tests, these produce test results that can be kept to track the progress in solving the error, and customer feedback can also be kept in the form of logs of issues encountered.

\section{Problem Correction}
Depending on the type of problem, a solution for the problem may have already been devised (for example in the risk assessment documents when the problem has already been identified as a risk). Other problems, such as a bug in the code, have methodologies already defined. For example, to solve a bug, the following appraoch is most often than not used:
\begin{enumerate}
    \item Locate the error using test results
    \item Modify the specification to design an error repair
    \item Repair the error
    \item Retest the program with the same test cases that produced the error
\end{enumerate}