\chapter{Code Of Ethics}
Like other engineering disciplines, software engineering is carried out within a social and legal framework that limits the freedom of people working in that area. As a software engineer it is imperative to understand and accept that the job involves wider responsibilities than simply the application of technical skills. It is also very important to behave in an ethical and morally responsible way in order to be respected as a professional engineer.

Besides notmal standards of honesty and integritym a software engineer should not use their skills and abilities to behave in a dishonest way or in a way that will bring disrepute to the software engineering profession. However, there are areas where standards of acceptable behaviour are not bound by laws but by the most tenous notion of professional responsibility. Some of these are:
\begin{itemize}
    \item Confidentiality: Confidentiality of the employers or clients should be normally respected regardless  of whether or not a formal confidentiality agreement has been signed.
    \item Competence: The level of competence of a software engineer should not be misrepresented by the individual, meaning that an engineer should not knowingly acceot work that is outside of their competence.
    \item Intellectual property rights: A software engineer should be aware of their local laws governing the use of intellectual property such as patents and copyright. They should also be careful to ensure that the intellectual property of employers and clients is protected.
    \item Computer misuse: A software engineer should not use their technical skills to misuse other people's computers. Computer misuse ranges from relatively trivial (playing a game on a employer's machine) to extremely serious (dissemination of viruses or other malware).
\end{itemize}

Professional societies and institutions have an important role to play in setting ethical standards, Organizations such as the ACM, the IEEE (Institute of Electronic Engineers), and the British Computer Society publish a code of professional conduct or code of ethics. Members of these organizations undertake to follow that code when they sign up for membership. These codes of conduct are generally concerned with fundamental ethical behaviour.

Professional associations, notably the ACM and the IEEE, have cooperated to produce a joint code of ethics and professional practice. This code exists in both a short form and a longer form that adds detail and substance to the shorter form. Here is the shorter form of the ACM\/IEEE Code of Ethics without the preamble:
\begin{enumerate}
    \item PUBLIC - Software engineers shall act consistently with the public interest.
    \item CLIENT AND EMPLOYER - Software engineers shall act in a manner that is in the best interest of their client and employer consistent with the public interest.
    \item PRODUCT - Software engineers shall ensure that their products and related modifications meet the highest professional standards possible.
    \item JUDGEMENT - Software engineers shall maintain integrity and independence in their professional judgement.
    \item MANAGEMENT - Software engineering managers and leaders shall subscribe to and promote an ethical approach to the management of software development and maintenance.
    \item PROFESSION - Software engineers shall advance the integrity and reputation of the profession consistent with the public interest.
    \item COLLEAGUES - Software engineers shall be fair to and supportive of their colleagues.
    \item SELF - Software engineers shall participate in lifelong learning regarding the practice of their profession and shall promote an ehical approach to the practice of the profession.
\end{enumerate}

The general area of ethics and profession responsibility is increasingly important as software-intensive systems prevade every aspect of work and everyday life. It is therefore very important that every software engineer upholds to them whenever working their job.